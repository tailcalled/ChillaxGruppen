\documentclass[a4paper]{article}

\usepackage{amsmath}
\usepackage{amsthm}
\usepackage{amsfonts}
\usepackage[utf8]{inputenc}
\usepackage{csquotes}
\usepackage{listings}
\usepackage{graphicx}
\usepackage{ifthen}
\usepackage{xspace}
\usepackage{hyperref}
\usepackage{mathtools}
\usepackage{tikz}
\usepackage{caption}
\usepackage{subcaption}

\DeclarePairedDelimiter\ceil{\lceil}{\rceil}
\DeclarePairedDelimiter\floor{\lfloor}{\rfloor}

\newcommand{\pgap}{~\\\noindent}

\newcommand{\Fsh}{{F$\sharp$}\xspace}

\newcommand{\col}[2]{{\begin{pmatrix} #1 \\ #2 \end{pmatrix}}}
\newcommand{\mmod}{\text{ mod }}

\newtheorem{theorem}{Theorem}
\newtheorem{lemma}{Lemma}
\newtheorem{definition}{Definition}

\lstset{basicstyle=\small\ttfamily,frame=leftline,numbers=left,xleftmargin=0.7cm,basewidth=0.14cm}
\lstset{rangeprefix=(*, rangesuffix=*),includerangemarker=false}

\lstnewenvironment{Code}{}{}
\newcommand{\codeInput}[2][]{\ifthenelse{\equal{#1}{}}{\lstinputlisting[title=#2]{#2}}{\lstinputlisting[linerange=#1-end,title=#2\ -\ #1]{#2}}}
\newcommand{\code}{\lstinline}

\usepackage{amssymb}

\title{DMA 12g}
\author{Carl Dybdahl, Patrick Hartvigsen, Emil Chr. Søderblom}

\begin{document}

\maketitle

\section*{Part 1}

\subsection*{(a)}

We need to show that given a path \(e_n\) with \(0 \le n < k\), we can rewrite \(-\log(w(e_0)\cdot w(e_1)\cdot \dots \cdot w(e_n))\) as a sum.

To do this, we exploit the rule \(\log(a\cdot b) = \log(a) + \log(b)\). We reason as follows:

\[
\begin{aligned}
-\log(w(e_0)\cdot w(e_1)\cdot \dots \cdot w(e_n)) &= -\log(w(e_0)) - \log(w(e_1)\cdot \dots \cdot w(e_n)) \\
&=  -\log(w(e_0)) - \log(w(e_1)) - \log(\dots \cdot w(e_n)) \\
&= \dots \\
&= -\log(w(e_0)) - \log(w(e_1)) - \dots - \log(w(e_n))
\end{aligned}
\]
\subsection*{(b)}

To find the most probable path, we wish to maximize \(\prod\limits_{0 \le n < k} w(e_n)\). This is equivalent to minimizing \(- \log\prod\limits_{0 \le n < k} w(e_n)\), which we have just shown is the same as \(\sum\limits_{0 \le n < k} - \log(w(e_n))\). This is equivalent to finding the shortests path with a weight function \(w'(e) = -\log(w(e))\). We use Dijkstra's algorithm for this:

\codeInput{pathfind.pseudocode}

\section*{Part 2}

\subsection*{(a)}

\(\delta_{BFS}(s, v)\) represents the length of the shortest path from \(s\) to \(v\), whereas \(\delta_{SP}(s, v)\) represents the cost of the lowest-cost path from \(s\) to \(v\). However, when \(w(e) = 1\), the cost of a path and its length coincides, and as such \(\delta_{BFS}\) and \(\delta_{SP}\) will be equal.

\subsection*{(b)}

A node is black during BFS if it has been visited. This means that the processing for that node is done.

The set \(S\) represents the nodes for which the shortest distance has been determined. Similarly to a node in BFS being black, a node being in \(S\) means that the processing for that node is done.

\subsection*{(c)}

The loop invariant states that at the start of each iteration of the while loop, \code|v.d = |\(\delta\)\code|(s, v)| for each vertex \code|v|. When Dijkstra's algorithm visits a node \code|u|, it has already visited all nodes \code|w| with a distance \(\delta(s, w)\) less than \(\delta(s, u)\). This means that any node in \(V\) with distance less than \(\delta(s, u)\) is already in \(S\), and so the nodes \code|v| in \(V \setminus S\) must have a distance greater than or equal to that of \code|u| from \code|s|.
\end{document}	