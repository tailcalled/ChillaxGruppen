\documentclass[a4paper]{article}

\usepackage{amsmath}
\usepackage{amsthm}
\usepackage{amsfonts}
\usepackage[utf8]{inputenc}
\usepackage{csquotes}
\usepackage{listings}
\usepackage{graphicx}
\usepackage{ifthen}
\usepackage{xspace}
\usepackage{hyperref}
\usepackage{mathtools}
\usepackage{tikz}

\DeclarePairedDelimiter\ceil{\lceil}{\rceil}
\DeclarePairedDelimiter\floor{\lfloor}{\rfloor}

\newcommand{\Fsh}{{F$\sharp$}\xspace}

\newcommand{\col}[2]{{\begin{pmatrix} #1 \\ #2 \end{pmatrix}}}
\newcommand{\mmod}{\text{ mod }}

\newtheorem{theorem}{Theorem}
\newtheorem{lemma}{Lemma}
\newtheorem{definition}{Definition}

\lstset{basicstyle=\small\ttfamily,frame=leftline,numbers=left,xleftmargin=0.7cm,basewidth=0.14cm}
\lstset{rangeprefix=(*, rangesuffix=*),includerangemarker=false}

\lstnewenvironment{Code}{}{}
\newcommand{\codeInput}[2][]{\ifthenelse{\equal{#1}{}}{\lstinputlisting[title=#2]{#2}}{\lstinputlisting[linerange=#1-end,title=#2\ -\ #1]{#2}}}
\newcommand{\code}{\lstinline}


\title{DMA Ugeopgave 4}
\author{Carl Dybdahl, Patrick Hartvigsen, Emil Søderblom}
\date{\today}

\begin{document}

\maketitle

\section{Del 1}

\subsection{a)}

Et objekt i listen \code|S| skal indeholde 3 attributter, \code|pre|, \code|key|, og \code|next|. \code|pre| skal pege på den tidligere knude, hvor \code|next| skal pege på den næste knude, og \code|key| skal være elementet ved knudens position. Hvis det er det første objekt i listen skal \code|pre = NIL| og hvis det er det sidste skal \code|next = NIL|.

Pseudokode:

\begin{Code}
def singleton(z):
	return new [ prev = NIL; next = NIL; key = z ]
\end{Code}

\subsection{b)}

\begin{Code}
def F(S, z):
	var prev = NIL
	var target = S
	while (target != NIL && target.key < z):
		prev = target
		target = target.next
	var node = singleton(z)
	if (target == NIL):
		prev.next = node
		node.prev = prev
	else:
		if (target.prev != NIL):
			target.prev.next = node
			node.prev = target.prev
		target.prev = node
		node.next = target
\end{Code}

\subsection{c)}

Hvis \code|z| skal placeres sidst i listen er funktionen nødt til at iterere gennem hele listen. Dette tager $\Theta(n)$ tid.

\subsection{d)}

Worst-case tilfældet er, at vi ved hver indsættelse skal iterere gennem hele listen - hvis dette er nødvendigt kommer vi til at bruge $\Theta(n^2)$, da vi $n$ gange skal iterere gennem en liste med gennemsnitlig størrelse $n/2$. Dette sker hvis vi hver gang skal indsætte et element der er større end alle de andre, altså hvis vi indsætte elementerne i sorteret rækkefølge.

I visse tilfælde, for eksempel hvis elementerne bliver indsat i omvendt sorteret rækkefølge, kan vi gøre det hurtigere end $\Theta(n^2)$, men det kan ikke blive langsommere. Dermed er kørselstiden $O(n^2)$ på sortering af lister med denne metode.

\section{Del 2}

\subsection{a)}

Vi laver en liste \code|B| og en løkke der kører gennem hver underliste $l_i$ i \code|S|. Denne løkke itererer $k$ gange, og hver gang indsætter vi i \code|B| en knude med en pejer på det tilsvarende element i \code|S|. Efter vi har indsat knuden gennemløber vi en anden løkke $k$ gange, hvor vi bevæger os frem til næste underliste.

\subsection{b)}

Da \code|B| har en længde på $\sqrt{n}$ vil det tage $\Theta(\sqrt{n})$.

\subsection{c)}

\begin{Code}
def G (S,B,x): 
	var prev = NIL
	var target = B
	while (target != NIL && target.pa.key < z)):
		prev = target
		target = target.next
	if (target == NIL):
		F(prev.pa, x)
	elseif (prev == NIL):
		var node = new [
			prev = NIL,
			next = target.pa,
			key = x
		]
		target.pa.prev = node
	else
		F(prev.pa, x)
\end{Code}

\subsection{d)}

Hvis en sorteret indsættelse kan gøres på $O(\sqrt{n})$ tid, kan man bare indsætte hvert element der skal sorteres, hvilket giver en samlet kørselstid på $O(n \sqrt{n})$. Dog skal man sørge for intelligent at vedligeholde \code|B|-listen, for ellers kan man ikke gentagende gange indsætte på $O(\sqrt{n})$ tid. 

\end{document}