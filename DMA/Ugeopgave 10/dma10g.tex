\documentclass[a4paper]{article}

\usepackage{amsmath}
\usepackage{amsthm}
\usepackage{amsfonts}
\usepackage[utf8]{inputenc}
\usepackage{csquotes}
\usepackage{listings}
\usepackage{graphicx}
\usepackage{ifthen}
\usepackage{xspace}
\usepackage{hyperref}
\usepackage{mathtools}
\usepackage{tikz}

\DeclarePairedDelimiter\ceil{\lceil}{\rceil}
\DeclarePairedDelimiter\floor{\lfloor}{\rfloor}

\newcommand{\Fsh}{{F$\sharp$}\xspace}

\newcommand{\col}[2]{{\begin{pmatrix} #1 \\ #2 \end{pmatrix}}}
\newcommand{\mmod}{\text{ mod }}

\newtheorem{theorem}{Theorem}
\newtheorem{lemma}{Lemma}
\newtheorem{definition}{Definition}

\lstset{basicstyle=\small\ttfamily,frame=leftline,numbers=left,xleftmargin=0.7cm,basewidth=0.14cm}
\lstset{rangeprefix=(*, rangesuffix=*),includerangemarker=false}

\lstnewenvironment{Code}{}{}
\newcommand{\codeInput}[2][]{\ifthenelse{\equal{#1}{}}{\lstinputlisting[title=#2]{#2}}{\lstinputlisting[linerange=#1-end,title=#2\ -\ #1]{#2}}}
\newcommand{\code}{\lstinline}

\usepackage{amssymb}

\title{DMA 9i}
\author{Carl Dybdahl, Patrick Hartvigsen, Emil Chr. Søderblom}

\begin{document}

\maketitle

\section*{Part 1}

We are given 3 linear homogeneous relation of degree 2 describing the run time of 3 different algorithms.

\begin{equation} \label{eq1} u_{n} = u_{n-1} + 4 \cdot u_{n-2} \end{equation}
\begin{equation} \label{eq2} u_{n} = 2 \cdot u_{n-1} + 3 \cdot u_{n-2} \end{equation}
\begin{equation} \label{eq3} u_{n} = 9 \cdot u_{n-2} \end{equation}

\subsection*{a)}
For each algoritm we have been asked to find the characteristic equation and its roots.

The characteristic equation for equation \ref{eq1} is
\(x^{2} - x - 4 = 0\).
To find the roots of this equation we use the quadratic formula
\(x=\frac{-b \pm \sqrt{b^2-4 \cdot a \cdot c}}{2 \cdot a} \)
, which yields
\(x_1 = 1/2 \cdot (1 - \sqrt{17})\) , \(x_2 = 1/2 \cdot (1 + \sqrt{17}) \).

The characteristic equation for equation \ref{eq2} is
\(x^{2} - 2 \cdot x - 3 = 0\).
The roots of this are
\(x_1 = -1\) , \(x_2 = 3\).

The characteristic equation for equation \ref{eq2} is
\(x^{2} - 0 \cdot x - 9 = 0\).
The roots of this are
\(x_1 = 3\), \(x_2 = -3\).

\subsection*{b)}
The task is to find constants \(s_1\), \(s_2\), \(s_3\) for each algorithm and show the run time is \(\Theta (s_1^2)\) for algorithm \ref{eq1}, \(\Theta (s_2^2)\) for algorithm \ref{eq2}, \(\Theta (s_3^2)\)  for algorithm \ref{eq3}.

The a linear homogeneous recurrence relation of degree 2 can be solved using its characteristic equation's roots.

In this case all equations have 2 roots, we can therefore insert the roots in \(a_n=u \cdot s_1^n + v \cdot s_2^n\) as \(s_1\) and \(s_2\) and get an function for the recursion equation.

Formula for \ref{eq1} is:
\[a_n=u \cdot (1/2 \cdot (1 - \sqrt{17}))^n + v \cdot (1/2 \cdot (1 + \sqrt{17}))^n\]

Because u and v is constants we have that \(u \cdot s_1^n + v \cdot s_2^n = \Theta (s_1^n + s_2^n)\). \(s_2\) is bigger than \(s_1\) which means that \(\Theta (s_1^n + s_2^n) = \Theta (s_2^n)\) and therefore the runtime is \[\Theta ((1/2 \cdot (1 + \sqrt{17})^n)\]


Formula for \ref{eq2} is:
\[a_n=u \cdot (-1)^n + v \cdot 3^n\]
The same argument as above applyes here and we get.
\[u \cdot (-1)^n + v \cdot 3^n = \Theta ((-1)^n + 3^n) = \Theta (3^n)\]


Formula for \ref{eq3} is:
\[a_n=u \cdot (-3)^n + v \cdot 3^n\]
The same argument as above applyes here and we get.
\[u \cdot 3^n + v \cdot (-3)^n = \Theta 3^n + (-3)^n) = \Theta (3^n)\]

\subsection*{c)}
The algorithm with the smallest runtime should go in further development and that is equation \ref{eq1}.

\section*{Part 2}

\subsection*{a)}

We will prove the statement by induction on k.
In the base case, \(k=1\), we have to consider two values of n (1 and 2), because \(n\leq2^1=2\). When we calculate the statement for \(k=1\) we can see that we have to prove that they are both less than or equal to 13:
\[
\begin{aligned}
  a_n & \leq 3 \cdot 1 \cdot 2^1 + 4 \cdot 2^1 - 1\\
  & = 13
\end{aligned}
\]
We can see that they are both less than or equal to 13:
\[
a_1 = 3 \leq 13
\]\[
\begin{aligned}
  a_2 & = a_1 + a_1 + 3 \cdot 2 + 1 \\
  & = 3 + 3 + 6 + 1 \\
  & \leq 13
\end{aligned}
\]
Thus we have that \(a_1 \leq 13\) and \(a_2 \leq 13\), and the statement is therefore true for the base case.

Next we have the inductive case, \(k=j+1\).
We assume that \(n \leq 2^j \implies a_n \leq 3 \cdot j2^j + 4 \cdot 2^j - 1\), and we must now prove \(n \leq 2^{j+1} \implies a_n \leq 3 \cdot (j+1) \cdot 2^{j+1} + 4 \cdot 2^{j+1} - 1\). We have that \(a_1 = 3\), so the following is for \(n \geq 2\):
\[
\begin{aligned}
  a_n & = a_{\floor*{n/2}} + a_{\ceil*{n/2}} + 3n + 1 \\
  & \leq 3 \cdot j \cdot 2^j + 4 \cdot 2^j - 1 + a_{\ceil*{n/2}} + 3n + 1 \\
  & \leq 3 \cdot j \cdot 2^j + 4 \cdot 2^j - 1 + 3 \cdot j \cdot 2^j + 4 \cdot 2^j - 1 + 3n + 1 \\
  & = 2(3 \cdot j \cdot 2^j + 4 \cdot 2^j - 1) + 3n + 1 \\
  & = 6 \cdot j \cdot 2^j + 8 \cdot 2^j - 2 + 3n + 1 \\
  & = 6 \cdot j \cdot 2^j + 8 \cdot 2^j + 3n - 1 \\
  & = 3 \cdot j \cdot 2^{j+1} + 4 \cdot 2^{j+1} + 3n - 1 \\
  & \leq 3 \cdot j \cdot 2^{j+1} + 4 \cdot 2^{j+1} + 3 \cdot 2^{j+1} - 1 \\
  & = 3 \cdot (j+1) \cdot 2^{j+1} + 4 \cdot 2^{j+1} - 1
\end{aligned}
\]

\subsection*{b)}

To see that \(a_n = O(n \log n)\), note that \(n \le 2^{\ceil*{\log_2 n}}\), and thus we have \(a_n \leq 3 \cdot \ceil*{\log_2 n} \cdot 2^{\ceil*{\log_2 n}} + 4 \cdot 2^{\ceil*{\log_2 n}} - 1\). As \(2^{\ceil*{\log_2 n}} = \Theta(n)\) and \(\ceil*{\log_2 n} = \Theta(\log n)\), we have \(3 \cdot \ceil*{\log_2 n} \cdot 2^{\ceil*{\log_2 n}} + 4 \cdot 2^{\ceil*{\log_2 n}} - 1 = \Theta(\log n \cdot n + 2^{\ceil*{\log_2 n}}) = \Theta(n \log n)\).

\end{document}	
