\documentclass[a4paper]{article}

\usepackage{amsmath}
\usepackage{amsthm}
\usepackage{amsfonts}
\usepackage[utf8]{inputenc}
\usepackage{csquotes}
\usepackage{listings}
\usepackage{graphicx}
\usepackage{ifthen}
\usepackage{xspace}
\usepackage{hyperref}
\usepackage{mathtools}
\usepackage{tikz}

\DeclarePairedDelimiter\ceil{\lceil}{\rceil}
\DeclarePairedDelimiter\floor{\lfloor}{\rfloor}

\newcommand{\Fsh}{{F$\sharp$}\xspace}

\newcommand{\col}[2]{{\begin{pmatrix} #1 \\ #2 \end{pmatrix}}}
\newcommand{\mmod}{\text{ mod }}

\newtheorem{theorem}{Theorem}
\newtheorem{lemma}{Lemma}
\newtheorem{definition}{Definition}

\lstset{basicstyle=\small\ttfamily,frame=leftline,numbers=left,xleftmargin=0.7cm,basewidth=0.14cm}
\lstset{rangeprefix=(*, rangesuffix=*),includerangemarker=false}

\lstnewenvironment{Code}{}{}
\newcommand{\codeInput}[2][]{\ifthenelse{\equal{#1}{}}{\lstinputlisting[title=#2]{#2}}{\lstinputlisting[linerange=#1-end,title=#2\ -\ #1]{#2}}}
\newcommand{\code}{\lstinline}

\usepackage{amssymb}

\title{DMA 9i}
\author{HELLO}

\begin{document}

\maketitle

\section*{Part 1}

We are given 3 linear homogeneous relation of degree 2 describing the run time of 3 different algorithms.

\begin{equation} \label{eq1} u_{n} = u_{n-1} + 4 \cdot u_{n-2} \end{equation}
\begin{equation} \label{eq2} u_{n} = 2 \cdot u_{n-1} + 3 \cdot u_{n-2} \end{equation}
\begin{equation} \label{eq3} u_{n} = 9 \cdot u_{n-2} \end{equation}
\subsection*{a}
For each algoritm we have been asked to find the characteristic equation and its roots.

The characteristic equation for equation \ref{eq1} is:
\[x^{2} - x - 4 = 0\]
To find the roots of this equation we use the quadratic formula
\(x=\frac{-b \pm \sqrt{b^2-4 \cdot a \cdot c}}{2 \cdot a} \)
, which yields
\(x_1 = 1/2 \cdot (1 - \sqrt{17})\) , \(x_2 = 1/2 \cdot (1 + \sqrt{17}) \)


The characteristic equation for equation \ref{eq2} is:
\[x^{2} - 2 \cdot x - 3 = 0\]
The roots of this is
\(x_1 = -1\) , \(x_2 = 3\)


The characteristic equation for equation \ref{eq2} is:
\[x^{2} - 0 \cdot x - 9 = 0\]
The roots of this is
\(x_1 = 3\), \(x_2 = -3\)

\subsection*{b}
Find constanst $s_1$, $s_2$, $s_3$ for each algorithm and show the run time is $\Theta (s_1^2)$ for \ref{eq1}, $\Theta (s_2^2)$ for \ref{eq2}, $\Theta (s_3^2)$  for \ref{eq3}.

The formula for a linear homogeneous relation of degree 2, can be made using its characteristic equations roots.
In this case all equations have 2 roots, we can therefore insert the roots in \(a_n=u \cdot s_1^n + v \cdot s_2^n\) as $s_1$ and $s_2$ and get an function for the recursion equation.

Formula for \ref{eq1} is:
\[a_n=u \cdot (1/2 \cdot (1 - \sqrt{17}))^n + v \cdot (1/2 \cdot (1 + \sqrt{17}))^n\]

Because u and v is constants we have that \(u \cdot s_1^n + v \cdot s_2^n = \Theta (s_1^n + s_2^n)\). $s_2$ is bigger than $s_1$ which means that \(\Theta (s_1^n + s_2^n) = \Theta (s_2^n)\) and therefore the runtime is \[\Theta ((1/2 \cdot (1 + \sqrt{17})^n)\]


Formula for \ref{eq2} is:
\[a_n=u \cdot (-1)^n + v \cdot 3^n\]
The same argument as above applyes here and we get.
\[u \cdot (-1)^n + v \cdot 3^n = \Theta ((-1)^n + 3^n) = \Theta (3^n)\]


Formula for \ref{eq3} is:
\[a_n=u \cdot (-3)^n + v \cdot 3^n\]
The same argument as above applyes here and we get.
\[u \cdot 3^n + v \cdot (-3)^n = \Theta 3^n + (-3)^n) = \Theta (3^n)\]

\subsection*{c}
The algorithm with the smallest runtime should go in further development and that is equation \ref{eq1}.




\end{document}	