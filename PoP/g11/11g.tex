\documentclass[a4paper]{article}

\usepackage{amsmath}
\usepackage{amsthm}
\usepackage{amsfonts}
\usepackage[utf8]{inputenc}
\usepackage{csquotes}
\usepackage{listings}
\usepackage{graphicx}
\usepackage{ifthen}
\usepackage{xspace}
\usepackage{hyperref}
\usepackage{mathtools}
\usepackage{tikz}
\usepackage{caption}
\usepackage{subcaption}

\DeclarePairedDelimiter\ceil{\lceil}{\rceil}
\DeclarePairedDelimiter\floor{\lfloor}{\rfloor}

\newcommand{\pgap}{~\\\noindent}

\newcommand{\Fsh}{{F$\sharp$}\xspace}

\newcommand{\col}[2]{{\begin{pmatrix} #1 \\ #2 \end{pmatrix}}}
\newcommand{\mmod}{\text{ mod }}

\newtheorem{theorem}{Theorem}
\newtheorem{lemma}{Lemma}
\newtheorem{definition}{Definition}

\lstset{basicstyle=\small\ttfamily,frame=leftline,numbers=left,xleftmargin=0.7cm,basewidth=0.14cm}
\lstset{rangeprefix=(*, rangesuffix=*),includerangemarker=false}

\lstnewenvironment{Code}{}{}
\newcommand{\codeInput}[2][]{\ifthenelse{\equal{#1}{}}{\lstinputlisting[title=#2]{#2}}{\lstinputlisting[linerange=#1-end,title=#2\ -\ #1]{#2}}}
\newcommand{\code}{\lstinline}


\title{POP 11g}
\author{Patrick Hartvigsen, Carl Dybdahl, Emil Søderblom}

\begin{document}

\maketitle

We have made an visual simulation of our solar system and compared the simulated data with the data from JPL Ephemeris.

\section{Design and Architecture}

\subsection{Architecture}

Figure \ref{uml} shows the UML diagram we made for this task. 

\begin{figure}[!ht]
\centering
\includegraphics{uml.png}
\caption{The UML diagram for our architecture.}
\label{uml}
\end{figure}

To easily be able to calculate with 3d-vectore we made an vector class called VE with som generic functions.
To be able to store and construct planets, we made an Planet class where its posistion and velocity are represented as vectors. We decided to make the attributes constants because of that to update a planets posistion and velocity, we contructed and new instance of that planet with the same mass and name but new posistion and velocity.

The PlanetsForm class inherit System.Form and initilizes a Form we uses to display oure solar system.


To find the posistion of the planets in a solar system, we made made a class called \code|SolarSystem| that stores a list of \code|Planet|. This class can simulates the planets movements in the solar system by calculating the planets posistion and velocity after a given time.


\subsection{Algorithms}



\section{Comparison with JPL Ephemeris}

\section{Defects}

\end{document}	