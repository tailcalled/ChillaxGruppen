\documentclass[a4paper]{article}

\usepackage{amsmath}
\usepackage{amsthm}
\usepackage{amsfonts}
\usepackage[utf8]{inputenc}
\usepackage{csquotes}
\usepackage{listings}
\usepackage{graphicx}
\usepackage{ifthen}
\usepackage{xspace}
\usepackage{hyperref}
\usepackage{mathtools}
\usepackage{tikz}
\usepackage{caption}
\usepackage{subcaption}

\DeclarePairedDelimiter\ceil{\lceil}{\rceil}
\DeclarePairedDelimiter\floor{\lfloor}{\rfloor}

\newcommand{\pgap}{~\\\noindent}

\newcommand{\Fsh}{{F$\sharp$}\xspace}

\newcommand{\col}[2]{{\begin{pmatrix} #1 \\ #2 \end{pmatrix}}}
\newcommand{\mmod}{\text{ mod }}

\newtheorem{theorem}{Theorem}
\newtheorem{lemma}{Lemma}
\newtheorem{definition}{Definition}

\lstset{basicstyle=\small\ttfamily,frame=leftline,numbers=left,xleftmargin=0.7cm,basewidth=0.14cm}
\lstset{rangeprefix=(*, rangesuffix=*),includerangemarker=false}

\lstnewenvironment{Code}{}{}
\newcommand{\codeInput}[2][]{\ifthenelse{\equal{#1}{}}{\lstinputlisting[title=#2]{#2}}{\lstinputlisting[linerange=#1-end,title=#2\ -\ #1]{#2}}}
\newcommand{\code}{\lstinline}


\title{INTER 1}
\author{Carl Dybdahl, Patrick Hartvigsen, Emil Chr. Søderblom}
\usepackage[danish]{babel}

\begin{document}

\maketitle

\section{Effektivetskrav} \label{efficiency}

Højst 5\% af hundrede førstegangsbrugere må fejle den følgende sekvens af tests:

\begin{itemize}
\item Brugerne skal på 10 minutter være i stand til at bestille en Opel Astra i Hamborg den 25. februar kl. 13:00, som skal afleveres i Berlin den 28. februar kl. 12:00.
\item Brugerne skal kunne afbestille denne bil på 5 minutter.
\item Brugerne, som nu har noget erfaring med websiden, skal på 5 minutter være i stand til at leje en Tesla Model S i Berlin den 1. marts kl. 12:00, som skal afleveres i Berlin den 1. marts kl. 17:00.
\item I denne test indeholder systemet på forhånd en bestilling for brugeren af en bil med Renault Grand Scenic i Frankfurt den 2. marts kl 13:00 som skal afleveres i Frankfurt den 3. marts kl 12:00. Brugeren skal på 10 minutter være i stand til at finde ud af biltype, biludlejeraddresse, opsamlingstidspunkt og afleveringstidspunkt.
\end{itemize}

\section{Tilfredshedskrav}

De hundrede testpersoner fra sektion \ref{efficiency} bliver spurgt om følgende udsagn, med svarmuligheder "Helt uenig", "Uenig", "Hverken enig eller uenig", "Enig", "Helt enig":

\begin{itemize}
\item Det er nemt at bestille en bil på hjemmesiden.
\item Det er let at overskue priserne.
\item Forsinkringens dækning er tydeligt fremvist og godt forklaret.
\item Hjemmesidens design er pænt.
\end{itemize}

Svarene bliver konverteret til en skala fra 1 til 5, hvor 1 er "Helt Uenig" og 5 er "Helt Enig". Gennemsnitssvaret for første udsagn skal være mindst 4, for andet og tredje udsagn mindst 3.5, og for fjerde udsagn mindst 3. Maksimalt 20\% af brugerne må have svaret "Helt Uenig" ved et spørgsmål.

\end{document}	